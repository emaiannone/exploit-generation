%! Author = emaia
%! Date = 10/03/20

WORK IN PROGRESS SECTION

In this section we explain the search algorithm used for the generation of exploit test cases.

\subsection{Background and assumptions}\label{subsec:assumptions}
- We will use the Whole Suite approach of Evosuite, that tries to produce a minimal test suite (minimze the size of the suite and the size of each test case) and maximizing the coverage

- Where we took the vulnerable constructs?
Talk about vulnerability dataset ~\cite{ponta2019msr_dataset}

- Evosuite let to instrument the libraries, so we can trace the information on the called constructs.

We chose to focus on the most simple kind of vulnerable construct, that is the vulnerable line, so the proposed
coverage criteria is a variation of a line coverage one.
When this approach will be mature enough, we will be extending on other kind of vulnerabilities.

\subsection{Fitness Goals}\label{subsec:goals}
- The goal (currently a single one, but to be extended to multiple goals)
- When a goal is considered covered

\subsection{Fitness Function}\label{subsec:fitness}
- Explain the formula

\subsection{Search Operators}\label{subsec:operators}
- To Be Implemented to improve performance

\subsection{Measuring the effort}\label{subsec:effort}
- To Be Implemented to measure how hard was/were a/some goals
%! Author = emaia
%! Date = 10/03/20

\documentclass[sigconf]{acmart}

%% \BibTeX command to typeset BibTeX logo in the docs
\AtBeginDocument{%
  \providecommand\BibTeX{{%
    \normalfont B\kern-0.5em{\scshape i\kern-0.25em b}\kern-0.8em\TeX}}}

\setcopyright{acmcopyright}
\copyrightyear{2018}
\acmYear{2018}
\acmDOI{10.1145/1122445.1122456}

\acmConference[Fisciano '20]{Fisciano '20: SD Software Dependability}
{June 12, 2020}{Fisciano, Italy}

%%
%% end of the preamble, start of the body of the document source.
\usepackage{verbatim}
\usepackage{cleveref}
\usepackage{amsmath}
\usepackage{url}

% define new commands
\newcommand\longvar[1]{\mathchardef\UrlBreakPenalty=100
    \mathchardef\UrlBigBreakPenalty=100\url{#1}}

\begin{document}

\title{Exploit Generation For Java Projects With Known OSS Vulnerabilities: An Extension For Evosuite}

\author{Emanuele Iannone}
\affiliation{%
  \institution{SeSa Lab - University of Salerno}
  \city{Fisciano (SA)}
  \country{Italy}
}
\email{e.iannone16@studenti.unisa.it}

%\author{Fabio Palomba}
%\affiliation{%
%    \institution{SeSa Lab - University of Salerno}
%    \city{Fisciano (SA)}
%    \country{Italy}
%}
%\email{fpalomba@unisa.it}
%
%\author{Dario Di Nucci}
%\affiliation{%
%    \institution{JADE Lab - University of Tilburg/JADS}
%    \city{'s-Hertogenbosch}
%    \country{The Netherlands}
%}
%\email{d.dinucci@uvt.nl}
%
%\author{Antonino Sabetta}
%\affiliation{%
%    \institution{SAP Security Research}
%    \city{Mougins}
%    \country{France}
%}
%\email{antonino.sabetta@sap.com}

%%
%% By default, the full list of authors will be used in the page
%% headers. Often, this list is too long, and will overlap
%% other information printed in the page headers. This command allows
%% the author to define a more concise list
%% of authors' names for this purpose.
\renewcommand{\shortauthors}{Iannone et al.}

\begin{abstract}
 \textbf{Funny abstract}
\end{abstract}

%%
%% The code below is generated by the tool at http://dl.acm.org/ccs.cfm.
%% Please copy and paste the code instead of the example below.
%%
\begin{CCSXML}
    <ccs2012>
        <concept>
            <concept_id>10011007.10011074.10011784</concept_id>
            <concept_desc>Software and its engineering~Search-based software engineering</concept_desc>
            <concept_significance>500</concept_significance>
        </concept>
    </ccs2012>
\end{CCSXML}
\ccsdesc[500]{Software and its engineering~Search-based software engineering}

%%
%% Keywords. The author(s) should pick words that accurately describe
%% the work being presented. Separate the keywords with commas.
\keywords{Search-Based Software Engineering, Software Testing, Software Vulnerabilities, Exploit Generation}

%%
%% This command processes the author and affiliation and title
%% information and builds the first part of the formatted document.
\begin{sloppypar}
\maketitle
\end{sloppypar}

\section{Introduction}\label{sec:introduction}
%! Author = emaia
%! Date = 10/03/20

% OSS diffusion
Almost all modern applications heavily rely on Open-Source Software (OSS) libraries.
Some estimates say that 80\% to 90\% of software products on the market includes some OSS components.
Other estimates say that the OSS components distribution depends on the kind of application: for commercial ones we have
a distribution of 35\% of total codebase, while for internal use application we reach the 75\% ~\cite{open_source}.

% OSS is vulnerable
The usage of OSS libraries lowers the development costs and time-to-market at the price of relying of something not fully validated.
As time goes by, more and more vulnerabilities of popular OSS libraries are being discovered and reported in Common
Vulnerabilities and Exposures (CVE) databases, like the National Vulnerability Database (NVD)~\cite{nvd}.
This growing trend made in 2013 to label \textit{`Using OSS components with known vulnerabilities'} as one of the
\textit{OWASP Top 10 Application Security Risks}.

% Detection tools: mitigation
To tackle this problem, numerous vulnerability detection tools are available,
either as OSS or as commercial products~\cite{owasp_dc, whitesource, ponta2018icsme_beyond}.
They analyze a given project to assess whether it relies on vulnerable OSS components.
Whenever a vulnerability is detected, the typical mitigation is asking to update the vulnerable library into a non-vulnerable version.
This proposed action is acceptable for software under development that is far its first release because all the necessary
adaptations in the application code can be performed as part of the normal development.
On the other hand, updating a library is a risky action for software in production or near its first release
because the update might force to modify most of the API calls (\textit{e.g.,} when updating to a different major release)
and to run regression tests, implying extra time and efforts.
This is better explained in~\cite{kula2017ese_updates} where the developers perceives the library updates as an extra workload and responsibility.

% Detection tools: detection
Some of these detection tools (\textbf{quali?}) use a naive but simple approach: including a certain vulnerable version of an OSS library
(\textit{e.g.,} having a certain dependency in \textit{pom.xml}) flags the entire project as potentially vulnerable.
The main drawback of this detection is that the vulnerable code might not be reachable at all, leading to numerous false positives.
For this reason, other tools (\textbf{quali? oltre Steady?}) relies on static source code analysis, by checking whether a certain vulnerable construct
(\textit{i.e.,} a general definition for any programming construct, \textit{e.g.,} a statement, a branch, a method, a class, a module, etc.)
is reachable in the Control Flow Graph;
however, this does not guarantee that the vulnerable construct can be exploited to carry an attack.
For example, that construct may be guarded by a good sanitization routine that prevents the construct to receive any malevolent input.
In order to expand these kind of analysis, other tools apply a dynamic analysis alongside the static one by relying on the available test suite
and on runtime information;
moreover, tools like Steady~\cite{ponta2018icsme_beyond} further expands the reachable constructs by applying an additional static analysis
on top of the results produced by the dynamic one (\textit{e.g.,} to reach constructs
called by the code loaded through Java Reflection~\cite{landman2017_reflection}).
Unfortunately, not all applications have access to a good test suite and/or runtime information (\textbf{qualcosa a supporto?}),
in these cases only a simple static analysis is applicable;
besides, the available tools do not assess the complexity of exploitability of the vulnerable constructs, so
it is hard to say whether a construct is actually exploitable and how it is hard to generate a concrete attack (\textbf{o malicious input?}).
Recently, there has been some initial efforts in search-based, for example, \citeauthor{jan2019_xmli}
proposed a novel genetic algorithm that tries to generate a set of malicious user inputs
for exploiting XML Injections~\cite{jan2019_xmli} in web applications.
However, this solution has some limitations: (i) it only works for a particular kind of vulnerability (\textit{i.e.,} XML Injections),
(ii) it only focuses on a precise set of vulnerable messages and (iii) it treats the SUT as a black-box, without
looking at the code nor at its API, so it does not know the concept of vulnerable construct.

% Proposal in short
In this work we propose a novel search-based technique that combines the efforts of \textsc{Eclipse Steady} and \textsc{EvoSuite}~\cite{fraser2013_evosuite} by
providing an additional coverage criteria for the latter to produce a test suite that exploits, at its best, the given vulnerabilities.
Whenever the generation fails to cover certain vulnerabilities, the technique gives some measures on the exploitability
of the given goals.

% Structure
Section~\ref{sec:approach} explains in detail the novel technique while giving some background information.
Section \textbf{CONTINUE}

% Metto questo?
%- Some tools relies on metadata associated to OSS libraries, so basing their analysis on a detailed description of the vulnerability (affected components, steps to reprocude, and so on).
%  - Problem: metadata not always clear or available.


\section{Background}\label{sec:background}
%! Author = emaia
%! Date = 14/03/20

In this section we briefly introduce the concept of genetic algorithm (GA)
and some well-known implementations in literature.

% SBSE
SBSE focuses on using search algorithms and techniques to solve Software Engineering-related problems~\cite{anand2013jss_survey};
in particular, Search-Based Software Testing (SBST) makes an extensive use of these techniques for solving problems
like Test Case Generation, Selection or Prioritization.
One of the most prominent problem in software testing is providing a minimal-cost test suite that maximizes a certain
criterion (\textit{e.g.,} covering the largest number of branches).
Generating test cases (often only the test input data) means using search algorithms guided by a
\textit{fitness function} that models a set of \textit{test goals} (\textit{a.k.a.,} test objectives).
The fitness function guides the search in the space of test cases to find those that best meet the test
goals;
this approach is highly generic and widely applicable is various forms.
There are many different kinds of search techniques, but in recent years the literature has been working on evolutionary approaches, in particular GAs.

\subsection{Genetic Algorithms}\label{subsec:genetic}
% GA: meta-heuristic
A GA is a particular type of a meta-heuristic optimisation algorithm (\textit{i.e.,} a general technique that can be applied to a broad range of search problems).
It starts with an initial random \textit{population} of \textit{individuals} (\textit{a.k.a.,} chromosomes),
which are candidate solutions for the given problem and whose genetic structure depends on the specific problem representation.
The population starts to evolve by passing through different generations (iterations) that continuously change its individuals
in order to produce the best solutions;
in a single iteration each individual is given to the fitness function that produces a score that represents its chance of surviving in the evolution.
The evolution consists of applying various operators on individuals, such as (i) \textit{selection}, that choose
the best individuals (the higher the fitness score, the higher the probability to be chosen) for the creation of the next generation of individuals;
(ii) \textit{crossover}, that applies some \textit{genes} interchange to each pair of selected \textit{parents} to generate their \textit{offsprings},
that will be added in the next generation;
(iii) \textit{mutation}, that randomly changes some genes of the newly-formed offsprings with a certain probability.
There may be different termination criteria: (i) the search budget (\textit{e.g.,} running time) may expire;
(ii) the population may reach the convergence (\textit{i.e.,} it has an individual with perfect fitness) or
(iii) the algorithm may stop making progresses, meaning that for some iterations the aggregate fitness score (related to the entire population) had no improvements.

% GA: Single vs Many
Typically, a fitness function is expressed in terms of a minimum function (\textit{i.e.,} the optimal value is the function's absolute minimum).
While common implementations use a single fitness function (\textit{a.k.a.,} single-objective optimisation), others prefers using multiple, often
conflicting, functions (\textit{a.k.a.,} many-objective optimisation), implying a redefinition of the concept of optimality that takes into
account multiple fitness scores --- collected into a \textit{fitness vector} --- and their trade-offs.
In many-objective problems we use the broader concepts of \textit{Pareto dominance} and \textit{Pareto optimality}~\cite{deb2005moo}.
%meaning that an individual is better
%than another one when it \textit{dominates} it, \textit{i.e.,} when it has at least one better fitness score than the other
%individual while the other scores are not worse than the other's.

\begin{definition}
    An individual $x$ \textit{dominates} another individual $y$ (also
    written $x \prec y$) if and only if the values of the fitness vector
    satisfy the following conditions:
    $\forall i \in \{1, \ldots, k\}, f_i(x) \leq f_i(y)$ and
    $\exists j \in \{1, \ldots, k\}, f_j(x) < f_j(y)$,
    where $k$ is the number of fitness functions.
\end{definition}

\begin{definition}
    An individual $x^*$ is \textit{Pareto optimal} if and only if it is not
    not dominated by any other individual in the space of all possible
    individuals (feasible region).
\end{definition}

\begin{definition}
    The \textit{Pareto front} is the set of all Pareto optimal individuals.
\end{definition}

% GA: in SBST
In Test Case Generation problem, a test suite is modeled as a population of evolving test cases (individuals),
which are, simply speaking, described as a list of statements and values (genes).
The test goals, that will define the fitness function(s), depend on the chosen \textit{coverage criterion}, that is commonly a
coverage measure from white-box testing, such as branch, line or mutation coverage.
The fitness function(s) will be based on how much the execution trace of a test case is close from the aforementioned goals;
for instance, for branch coverage the fitness is based on the number of control dependencies that separate the execution trace
from the target branch (\textit{approach level}) and on the variable values evaluated at the conditional expression where the execution
diverges from the target (\textit{branch distance})~\cite{panichella2018tse_dynamosa}.

\subsection{Notable Examples}\label{subsec:examples}
The various forms of fitness functions are not the only variants that can be applied on GAs;
indeed, there are numerous examples in literature of how malleable a GA is.

% MOSA: preference and archive
\citeauthor{panichella2015icst_mosa}~\cite{panichella2015icst_mosa} proposed MOSA (Many-Objective Sorting Algorithm),
which uses multiple fitness functions, one for each branch of the SUT (any other coverage criterion is valid),
to generate the best set of Pareto optimal test cases that minimize all the fitness functions.
In many-objective problems, however, the number of optimal individuals increases exponentially with the number of objectives,
making the search much more difficult (\textit{i.e.,} the convergence is hardly reached) and the algorithm to perform similarly to a random search one.
Thus, domain-specific knowledge is required to impose some \textit{preference criteria} among the set of non-dominated test cases,
for instance, in test case generation problem test cases that (i) cover (or are close to cover) uncovered targets and
(ii) are smaller than other competitors are preferred over others.
As consequence, the set of individuals candidate (for reproduction) is a subset of the whole Pareto front.
In addition to these preference criteria, MOSA adds the \textit{archive} technique, that consists of keeping a distinct
non-evolving population containing the test cases that satisfy previously uncovered targets;
after the final iteration, the test cases are picked from both the last population and the archive to form the final test suite.

% DynaMOSA: dynamic selection of targets
The main limitation of MOSA is that it treats all coverage goals as independent objectives, when, actually,
there exist structural dependencies among them that should be considered when deciding which one to optimise;
for example, a certain branch could be satisfied if all branches that holds a control dependency on it are already covered.
To overcome this limitation, the same main author of MOSA proposed DynaMOSA (Dynamic MOSA)~\cite{panichella2018tse_dynamosa}
that is able, in each iteration, to \textit{dynamically select targets} whose control dependency holders have already been covered in previous
iterations and to ignore the other targets.
This idea makes the search more effective and efficient in cases where there are a lot of dependencies among coverage goals.

% MIO: independent evolutionary algorithm
A quite different optimisation approach is with MIO (Many Independent Objectives)~\cite{arcuri2017lncs_mio}, an evolutionary algorithm
that works well for hundreds/thousands of independent (though not incompatible) goals.
It keeps one population for each goal --- called \textit{islands} --- that evolve independently from the others
(namely, the individuals of an island are only evaluated against the respective fitness function).
The reproduction mechanism is not based on genetic operators (this is why MIO is not classified as a GA), but rather on sampling individuals either
randomly or from other islands (\textit{i.e., migration}).
Whenever a goal is covered, its island stops evolving and only the (single) best individual survives;
at the end of the entire search, the set of best individuals forms the final solution.

% Co-evolutionary: COMIX
The MIO approach can be combined with GAs, creating the so-called class of \textit{co-evolutionary} algorithms,
that consists of a set of islands that evolve independently but using standard genetic operators with some periodical migrations (though with some variations).
An example of co-evolutionary algorithm is COMIX (Cooperative Co-evolutionary Algorithm for XMLi)~\cite{jan2019_xmli},
used for the automatic generation of malicious user inputs that exploit XML Injection vulnerabilities in web applications.
An important difference between COMIX and the other aforementioned algorithms is that the goals are not based on
the typical coverage measures from white-box testing, but, instead, they are based on a precise set of malicious XML messages, making the strategy totally black-box.

% Evosuite: Whole suite
\citeauthor{fraser2013_evosuite} in the context of \textsc{EvoSuite}~\cite{fraser2013_evosuite} proposed
the \textit{whole suite approach} that changes the point of view:
instead of evolving a population of test cases, it evolves a population of test suites and the single
fitness function models all the coverage goals at once.
The fitness score of a single test suite is an aggregation of the fitness `sub-scores' of each composing test case.
Moreover, the \textsc{EvoSuite}'s algorithm has a preference for smaller test suite, \textit{i.e.,} it tries to find the test suite
with the minimal number of total statements among all individuals.
\textsc{EvoSuite} tool supports multiple coverage criteria, such as branch, method, line, def-use, mutation, exceptions, etc.
and it is able to \textit{generate assertions} (\textit{i.e.,} test oracles) for each produced test case of the final test suite, too.


\section{The Novel Approach}\label{sec:approach}
%! Author = emaia
%! Date = 10/03/20

\textbf{WORK IN PROGRESS SECTION}

In this section we explain the novel search algorithm used for the generation of exploit test cases.

\subsection{Preliminary Assumptions}\label{subsec:assumptions}
% Dataset
\textbf{Should I explain the dataset here or in another section (maybe 4?) when presenting the validation?}
We chose to analyze the vulnerabilities of well-known OSS Java libraries and frameworks (\textit{e.g.,} Spring, Jenkins, Spark, Kafka).
Their vulnerabilities are publicly disclosed as CVE and collected in the NVD. In particular \citeauthor{ponta2019msr_dataset},
defined and made available a manually-curated dataset of fixes of these known
vulnerabilities~\cite{ponta2019msr_dataset}.
This dataset maps for each CVE entry the related repository URL and the signature of the fix commit
(\textit{i.e.,} the commit that officially patches the vulnerability).
So, the dataset has the form of a set of triples \textit{(CVE\_entry, repository\_url, fix\_commit\_hash)}, for example
(CVE-2017-4971, \url{https://github.com/spring-projects/spring-webflow}, 57f2ccb66946943fbf3b3f2165eac1c8eb6b1523).
Each fix commit is very different: some of them only apply changes on few lines in a single class, while others
consists of addition and refactoring of methods in various classes. (\textbf{I don't like this sentence})
We are interested in understanding the set of vulnerable constructs (\textit{i.e.,} the set of fixed constructs),
because we want to analyze the various forms of vulnerabilities to find a common representation that is suitable for the search algorithm.

% Assumptions
We chose to focus on what we consider the most simple type of vulnerable construct, that is the vulnerable line (\textit{i.e., }
a source code line that an attacker can exploit by giving it a certain malicious input), so the proposed coverage criterion is a
variation of the line coverage one.
When this approach will be mature enough, we will extend it to other types of vulnerabilities.

% Evosuite: why extend
We chose to implement this novel algorithm as an extension of \textsc{EvoSuite}~\cite{fraser2013_evosuite},
an automatic \textit{JUnit} tests generator for \textit{Java} classes by using a whole suite evolutionary approach
to derive the best possible test suites (\textit{i.e.,} covering all feasible goals and having the minimal number of total statements).
\textsc{EvoSuite} is a fully fledged GA framework, so it allows to (i) add new coverage criteria,
(ii) add new genetic operators (\textit{i.e.,} selection, crossover and mutation), (iii) reuse well-validated meta-heuristics and
(iv) reuse the instrumentation and program analysis tools, such as Control Flow Graphs, Call Graphs and Execution Traces.
It is possible to extend \textsc{EvoSuite} by adding some classes (related to the new coverage criteria and genetic operators),
and make small changes in others.
Sections~\ref{subsec:goals},~\ref{subsec:fitness} and~\ref{subsec:operators} provides additional details on the search algorithm.
In the context of our work we are not generating tests in the narrow sense (\textit{i.e.,} program execution that verifies a certain
expected behaviour) but `exploits cases', so we disabled the assert generation engine \textbf{Are we sure it can be disabled?}.

\subsection{Coverage Goals}\label{subsec:goals}
\textbf{I have to be sure about the descriptors}

Our approach is able to work on multiple coverage goals in a single run.
A coverage goal is defined from a triple of strings \textsc{(vulnerable\_class}, \textsc{vulnerable\_method},
\textsc{vulnerable\_line}) where \textsc{vulnerable\_class} is the fully-qualified name of the vulnerable class (\textit{e.g.,}
\sloppy\textit{\url{org.springframework.webflow.mvc.view.AbstractMvcView}}), \textsc{vulnerable\_method} is the name of the vulnerable
method concatenated with its descriptor (\textit{e.g.,}
\sloppy\textit{\url{addEmptyValueMapping(Lorg/springframework/binding/mapping/implDefaultMapper;Ljava/lang/String;Ljava/lang/Object;)V}})
and \textsc{vulnerable\_line} is the number of the vulnerable line in the source code.
The strings \textsc{vulnerable\_class} and \textsc{vulnerable\_method} are used to determine the right \textit{call context},
\textit{i.e.,} the list of method calls from the class under test (CUT) needed to reach the vulnerable method in the vulnerable class;
the string \textsc{vulnerable\_line} is used to retrieve the list of control dependencies on the vulnerable line
(\textit{i.e.,} predicates to be satisfied in order to certainly execute that line).

In other words, each coverage goal is composed of (i) a call context to reach a certain vulnerable method and
(ii) the list of control dependencies needed to execute vulnerable line.
An individual (test suite) covers a goal whenever it has a test case whose execution trace covers the call context and all
control dependencies in the list.

\textbf{Currently, I tried with only a single hard-coded goal per run.
The next steps will be (i) find a way to create the goal without hard-coding and (ii) try the approach with multiple goals}

\subsection{Fitness Function}\label{subsec:fitness}
- Explain the (primitive) formula

\subsection{Genetic Operators}\label{subsec:operators}
- To Be Implemented to improve performance

\subsection{Measuring the Effort}\label{subsec:effort}
- To Be Implemented to measure how hard was/were a/some goals


\textbf{Should I put a concrete example?}

\begin{acks}

\end{acks}

%%
%% The next two lines define the bibliography style to be used, and
%% the bibliography file.
\bibliographystyle{ACM-Reference-Format}
\bibliography{biblio}

%%
%% If your work has an appendix, this is the place to put it.
%% \appendix

%% \section{Research Methods}

%% \subsection{Part One}

%% \subsection{Part Two}

%% \section{Online Resources}

\end{document}
\endinput

%! Author = emaia
%! Date = 15/06/20

The execution of the 8 artificial examples let us discover something related to the behaviour of \textsc{EvoSuite}.

The running time and the optimality are heavily influenced by the complexity of reachability of the vulnerable
line, especially during the minimisation of the approach level ---
namely, the latter two vulnerabilities (CVE-2017-1000390 and CVE-2018-8718) are hard to exploit
because they involve the usage of \textbf{very complex objects} that must be put in a specific state
in order to pass some branches or avoid raising exceptions.
This make the artificial examples ineffective and highlights the urgency of switching to real projects for
more realistic investigations.

There are some cases where covering all control dependencies of the vulnerable line is not sufficient;
in fact, merely reaching its belonging block does not guarantee the fact that the line will be covered.
For instance, just consider a \textsc{throw} or \textsc{return} statement just before the target line, or
a method call that can potentially throw an uncontrolled exception.
To tackle this problem, \textbf{the coverage goal definition should be updated} to consider the actual execution of the
target line.

The whole suite approach --- enabled by default --- evolves entire test suites, and we have observed that most of them
ranges from one to dozens of test cases of variable length (about 2--25 lines each) and similar fitness scores,
making them not very useful.
Moreover, according to our goal's definition, we just need a single final exploit test case per goal,
and not an entire test suite.
The artificial examples are made to work on a single coverage goal only for the sake of simplicity, however,
it would be good to work on multiple goals (\textit{i.e.,} vulnerabilities) at a time,
that would be independent from each other.
All the aforementioned issues could be tackled by \textbf{the usage of a different meta-heuristic},
such as MIO, that may model the problem more accurately.

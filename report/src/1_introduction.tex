%! Author = emaia
%! Date = 10/03/20

% OSS diffusion
Almost all modern applications heavily rely on Open-Source Software (OSS) libraries.
Some estimates say that 80\% to 90\% of software products on the market includes some OSS components.
Other estimates say that the OSS components distribution depends on the kind of application: for commercial ones we have
a distribution of 35\% of total codebase, while for internal use application we reach the 75\% ~\cite{open_source}.

% OSS is vulnerable
The usage of OSS libraries lowers the development costs and time-to-market at the price of relying on something that is not fully validated.
As time goes by, more and more vulnerabilities of popular OSS libraries are being discovered and reported in Common
Vulnerabilities and Exposures (CVE) databases, such as the National Vulnerability Database (NVD)~\cite{nvd}.
This growing trend made in 2013 to label \textit{`Using OSS components with known vulnerabilities'} as one of the
\textit{OWASP Top 10 Application Security Risks}.

% Detection tools: mitigation
To tackle this problem, numerous vulnerability detection tools are available,
either as OSS or as commercial products~\cite{owasp_dc, whitesource, spotbugs, sonar, ponta2018icsme_beyond};
they analyze a given project to assess whether it relies on vulnerable OSS components and if
a vulnerability is detected, the commonly suggested mitigation is updating the vulnerable library into a non-vulnerable version.
This proposed action is acceptable for software still under development that is far from its first release because all the necessary
adaptations in the application code can be performed as part of the normal development.
On the other hand, updating a library is a risky action for software in production or near its first release
because the update may force to modify most of the API calls (\textit{e.g.,} when updating to a different major release)
and to run regression tests, implying extra time and efforts.
This is better explained in~\cite{kula2017ese_updates} where the developers perceive the library updates as an extra workload and responsibility, making them reluctant.

% Detection tools: detection
Some of these detection tools~\cite{owasp_dc, whitesource} use a simple approach: including a certain vulnerable version of an OSS library
(\textit{e.g.,} having a certain dependency in \textit{pom.xml} file) flags the entire project as potentially vulnerable.
The main drawback of this detection is that the vulnerable code may not be reachable at all, leading to numerous false positives.
For this reason, other tools~\cite{spotbugs, sonar, ponta2018icsme_beyond} rely on static source code analysis by checking whether a certain vulnerable construct
(\textit{i.e.,} a general definition for any programming construct, such as a statement, a branch, a method, a class, a module, etc.)
is reachable in the Control Flow Graph (CFG);
while this approach is more robust, it this does not guarantee that the vulnerable construct can be exploited to carry an attack.
For example, that construct may be guarded by a good sanitization routine that prevents the construct to receive any malevolent input.
In order to expand these kind of analysis, certain tools apply a dynamic analysis alongside the static one by relying on the available test suite
and on runtime information;
moreover, tools like \textsc{Eclipse Steady}~\cite{ponta2018icsme_beyond} further expands the reachable vulnerable
constructs by applying an additional static analysis on top of the results produced by the
dynamic one (\textit{e.g.,} to reach constructs called by the code loaded through Java Reflection~\cite{landman2017_reflection}).
Unfortunately, not all applications have access to a good test suite and/or runtime information,
in these cases only a simple static analysis is applicable;
moreover, the available tools do not assess the complexity of exploitability of the vulnerable constructs, so
it is hard to say whether a construct is actually exploitable and how it is hard to generate a concrete attack.
Recently, there has been some initial efforts in Search-Based Software Engineering (SBSE) field, for example,
\citeauthor{jan2019_xmli}~\cite{jan2019_xmli} proposed a novel genetic algorithm that tries to
generate a set of malicious user inputs for performing XML Injections in web applications.
However, this solution has some limitations: (i) it only works for a specific kind of vulnerability (namely, XML Injections),
(ii) it only focuses on a precise set of vulnerable messages and (iii) it treats the system under test (SUT) as a black-box, without
looking at the code or at its API, so it is not aware of the concept of vulnerable construct.

% Proposal in short
In this work we propose a novel search-based vulnerability assessment technique that combines the efforts
of \textsc{Eclipse Steady} and \textsc{EvoSuite}~\cite{fraser2013_evosuite} by providing an additional coverage
criterion for the latter to produce a test suite that exploits, at its best, the vulnerabilities discovered by the former.
Whenever the generation fails to cover certain vulnerabilities, the technique gives some measures on the
degree of exploitability of the given goals.

% Structure
Section~\ref{sec:background} provides some background information.
Section~\ref{sec:approach} makes some preliminary assumptions and explains in detail the novel technique.
Section~\ref{sec:example} shows some artificial example.
Section~\ref{sec:findings} highlights the main findings and implications.
Section~\ref{sec:conclusions} concludes the report by presenting the future objectives.
